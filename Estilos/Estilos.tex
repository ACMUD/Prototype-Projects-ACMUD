


% some color definitions
\definecolor{white}{RGB}{255,255,255}
\definecolor{blue}{RGB}{34,83,172}
\definecolor{gray}{RGB}{75,86,88}
\definecolor{lightblue}{RGB}{160,190,240}
\definecolor{lightgray}{RGB}{172,198,204}
\definecolor{extralightgray}{RGB}{228,237,239}
\definecolor{commentcolor}{rgb}{0,.5,0}
\definecolor{stringcolor}{rgb}{.62,.125,.94}
\definecolor{commentcolor}{rgb}{0,.5,0}
\definecolor{stringcolor}{rgb}{.62,.125,.94}
% COLORES personales---------------------------------------------------
    \definecolor{colortitulo}{RGB}{11,17,79} % 
    \definecolor{colordominante}{RGB}{11,17,79}
    \definecolor{colordominanteF}{RGB}{219,68,14}
    \definecolor{colordominanteD}{RGB}{137,46,55}
    \definecolor{mostaza}{RGB}{231,196,25}
    \definecolor{amarilloM}{RGB}{248,199,90}
    \definecolor{amarilloD}{RGB}{251,237,121}
    \definecolor{azulF}{rgb}{.0,.0,.3}
    \definecolor{grisD}{rgb}{.3,.3,.3}
    \definecolor{grisF}{rgb}{.6,.6,.6}
    \definecolor{grisamarillo}{RGB}{248,248,245} 
    \definecolor{miverde}{RGB}{44,162,67}
    \definecolor{verdep}{RGB}{166,206,58}
    \definecolor{verdencabezado}{RGB}{166,206,58}
    \definecolor{verdeF}{RGB}{5,92,8}
    \colorlet{mygray}{black!20}
    \newcommand{\verde}{\color{miverde}}
    \definecolor{AzulLT}{RGB}{120,133,246}
    \definecolor{numeSec}{RGB}{73, 9, 59}
% Fin COLORES persona


%%%% paquetes -------------

\usepackage{pstricks}
\usepackage{xparse}
\usepackage{tcolorbox} 
\tcbuselibrary{skins,breakable}                    % Librerías tcolorbox
\usetikzlibrary{positioning,shadows,backgrounds,calc}%
\usepackage{tikzpagenodes}
\usepackage{xargs}                                 % Comandos con opciones
% \usepackage[small,bf]{caption}
% \usepackage[breaklinks,colorlinks=true, pdfstartview=FitV, linkcolor=azulF, citecolor=azulF, urlcolor=azulF]{hyperref}
\usepackage{amsmath,amssymb,amsfonts,latexsym,cancel,stmaryrd}%
\usepackage[ruled,,vlined,lined,linesnumbered,algochapter]{algorithm2e}
\usepackage{multicol}
\usepackage{framed}
\usepackage{titletoc}
\usepackage{calc}
\usepackage{colortbl} 
\usepackage{tabularx}
\usepackage{fancyvrb}
\usepackage{array}
\usepackage{wasysym}
\usepackage{supertabular}
\usepackage{booktabs}
\usepackage[shortlabels]{enumitem}


% fin paqueyes ----

% Fuentes
%----------------------------------------------------------------------------------------
% Comandos para fuentes especiales
\newcommandx*{\fnte}[4][1=pag,2=9,3=n]{{\color{AzulLT}\fontfamily{#1}
\fontsize{#2}{1}\fontshape{#3}\selectfont{#4}}}

\newcommandx*{\fntb}[4][1=pag,2=9,3=n]{{\color{AzulLT}\fontfamily{#1}\fontsize{#2}{1}\fontseries{b}\fontshape{#3}\selectfont{#4}}}

\newcommandx*{\fntg}[4][1=pag,2=9,3=n]{{\color{AzulLT}\fontfamily{#1}\fontsize{#2}{1}\fontshape{#3}\selectfont{#4}}}

\newcommand{\fhv}[2]{{\fontfamily{pag}\fontsize{#1}{1}\selectfont{#2}}}

\newcommand{\fhvb}[2]{{\fontfamily{pag}\fontseries{b}\fontsize{#1}{1}\selectfont{#2}}}
% Fin fuentes------------------------------------------

%********************************** DISENO *************************************



%-------------------CONTENIDO -----------------------------------------------------
% \definecolor{mycolorA}{RGB}{0,133,202}  % 
% \definecolor{mycolorB}{RGB}{166,206,58} % 

% % patching of \tableofcontents to use sans serif font for the tile
% \patchcmd{\tableofcontents}{\contentsname}{\contentsname}{}{}
% % patching of \@part to typeset the part number inside a framed box in its own line
% % and to add color
% \makeatletter
% \patchcmd{\@part}
%   {\addcontentsline{toc}{part}{\thepart\hspace{1em}#1}}
%   {\addtocontents{toc}{\protect\addvspace{20pt}}
%     \addcontentsline{toc}{part}{\huge{\protect\color{colordominante}%
%       \setlength\fboxrule{2pt}\protect\Circle{%
%         \hfil\thepart\hfil%
%       }%
%     }\\[2ex]\color{colordominante}\sffamily\large#1}}{}{}
% \makeatother

% % this is the environment used to typeset the chapter entries in the ToC
% % it is a modification of the leftbar environment of the framed package
% \renewenvironment{leftbar}
%   {\def\FrameCommand{\hspace{6em}%
%     {\color{numeSec}\vrule width 2pt depth 6pt}\hspace{1em}}%
%     \MakeFramed{\parshape 1 0cm \dimexpr\textwidth-6em\relax\FrameRestore}\vskip2pt%
%   }
%  {\endMakeFramed}

% % using titletoc we redefine the ToC entries for parts, chapters, sections, and subsections
% \titlecontents{part}
%   [0em]{\centering}
%   {\contentslabel}
%   {}{}
% \titlecontents{chapter}
%   [0em]{\vspace*{2\baselineskip}}
%   {\parbox{4.5em}{%
%     \hfill\Huge\bfseries\color{mycolorA}\thecontentspage}%
%   \vspace*{-2.3\baselineskip}\leftbar{\fhvb{12}{\chaptername~\thecontentslabel}}\\}
%   {}{\endleftbar}
% \titlecontents{section}
%   [8.4em]
%   {\sffamily\contentslabel{3em}}{}{}
%   {\hspace{0.5em}\nobreak\itshape\color{mycolorA}\contentspage}
% \titlecontents{subsection}
%   [8.4em]
%   {\sffamily\contentslabel{3em}}{}{}  
%   {\hspace{0.5em}\nobreak\itshape\color{mycolorA}\contentspage}

% Fin Contenido 

%-------------------------------------------------------------------------------------
%	Numeración de las secciones -- en el margen
%----------------------------------------------------------------------------------------

\makeatletter
\renewcommand{\@seccntformat}[1]{\llap{\textcolor{AzulLT}{\csname the#1\endcsname}\hspace{1em}}}                    
\renewcommand{\section}{\@startsection{section}{1}{\z@}
{-4ex \@plus -1ex \@minus -.4ex}
{1ex \@plus.2ex }
{\normalfont\large\sffamily\bfseries}}
\renewcommand{\subsection}{\@startsection {subsection}{1}{\z@}
{-3ex \@plus -0.1ex \@minus -.4ex}
{0.5ex \@plus.2ex }
{\normalfont\sffamily\bfseries}}
\renewcommand{\subsubsection}{\@startsection {subsubsection}{3}{\z@}
{-2ex \@plus -0.1ex \@minus -.2ex}
{0.2ex \@plus.2ex }
{\normalfont\small\sffamily\bfseries}}                        
\renewcommand\paragraph{\@startsection{paragraph}{4}{\z@}
{-2ex \@plus-.2ex \@minus .2ex}
{0.1ex}
{\normalfont\small\sffamily\bfseries}}
\makeatother
% Fin numeración secciones

